\documentclass[12pt]{article}
\usepackage{wrapfig}
\usepackage{Foreman}
\usepackage{amsmath}
\title{Classifying Securities\vspace{-2ex}}
\author{Ivan Sepulveda \\ Faculty Advisor: Shivani Shukla}
\date{\vspace{-1.5ex}Spring 2019\vspace{-2ex}}
\usepackage[utf8]{inputenc}
\usepackage[english]{babel}
\usepackage{multicol}
\usepackage{float}
\usepackage{wrapfig}
\usepackage{booktabs}
\usepackage{graphicx}
\usepackage{subcaption}
\usepackage{amsmath}
\usepackage{siunitx}
\usepackage[english]{babel}
\usepackage[colorlinks]{hyperref}
\usepackage{enumitem}

\setlist[description]{leftmargin=\parindent,labelindent=\parindent}

\usepackage[table]{xcolor}% http://ctan.org/pkg/xcolor

%The next line causes a paragraph indent
\setlength{\parindent}{1cm} % Default is 15pt.
\captionsetup[subfigure]{textfont=normalfont,singlelinecheck=off,justification=raggedright}



\begin{document}
\maketitle
\hrule
\vspace{2ex}
Using the classification methods of Logistic Regression, Support Vector Machine, KNN, and K-means clustering, the goal of this project is to determine which financial variables are most relevant when seeking to resolve whether or not an individual stock's percent rise/fall surpassed the Standard and Poor 500's increase for the same time period.

\vspace{0.001cm}
\hrule


\section{Introduction}

\section{Theory}

\section{Features}

\section{Conclusion}

\section{Acknowledgements}
This work was supported by the University of San Francisco.

\begin{thebibliography}{9}
	\bibitem{a} 
	First Reference (2000).
	\bibitem{b} 
	Second Reference (2001).
	\bibitem{c} 
	Third Reference (2002).
	\bibitem{d}
	Forth Reference (2003).


\end{thebibliography}


\end{document}